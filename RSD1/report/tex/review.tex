%r
\section{Review of possible systems}
Washing machines facilities are use from traditional to the industrial. Main differences of those two areas are water usage, clothes amount and etc.

\begin{table}[htbp]
	\centering
    \begin{tabular}{ | p{3.5cm} | p{3.5cm} | p{3.5cm} | p{3.5cm} |}
    \hline
    \multicolumn{2}{|c|}{\textbf{Commercial}} & \multicolumn{2}{|c|}{\textbf{Non Commercial}} \\ \hline
    \multicolumn{2}{|l|}{Hotels, hospitals, elder homes and etc.} & \multicolumn{2}{|l|}{Rooms, dormitories and etc.} \\ \hline
    Advantages & Disadvantages & Advantages & Disadvantages \\ \hline
    High efficiency. Usage of the water can be decreased. (High amounts of clothes can be washed more efficiency than small amounts. For instance during one wash, washing machine utilizes the same amount of water. Higher amounts of clothes can be more optimally to wash then small. & Delays. (1. Maybe sometimes need to wait while laundry room wash your clothe. 2. Person needs move from home to laundry room wash his clothes. Both situations uses person time) & Cheaper, faster. (Sometimes faster to wash at home than to move somewhere in the city) & High usage of water. (Usage of the water is not optimal: During one wash not always possible find the max amount of clothes.) \\ \hline
    \end{tabular}	\caption{Advantages/disadvantages of }
	\label{tab:AdDis}
\end{table}